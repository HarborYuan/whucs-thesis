% 中文摘要

摘要是论文的缩影,是全文的高度概括和浓缩,便于读者了解全文得梗概。摘要也是扩大流通的媒介,有的情报人员根据论文摘要编制索引资料,也有的编入文摘刊物,这样流通范围会大大扩展。

论文摘要包含5个方面的内容:论文选题的背景(研究的目的);要解决的问题研究的内容);所使用的的研究方法;研究的结果;研究的主要结论及其意义。

摘要是一篇具有完整性和独立性的短文,不加评论和补充的解释。摘要篇幅不少于半页,3000字以内。特别注意:摘要中不要大篇幅地写背景、问题的重要性等内容(该部分不能超过摘要篇幅的四分之一)。

摘要的语言必须提纲携领,言简意赅,重点突出。中文摘要不要用第一人称语气,如“本文”、“我们”、“我们”等,而需要使用第三人称语气。英语摘要中提到本篇论文可以用要用this thesis 或this dissertation,但是不要用this paper。

注意摘要中不得出现“本文共有X章,第1章…,第2章…”之类的表述。

摘要中不要引用参考文献。
