% Chapter 2
\chapter{研究背景}

本章针对自己阅读的综述以及本人阅读的论文,归纳相关国内外研究现状,在阐述研究现状的时候一定要一步一步引出自己为什么做本文的工作。可以分节介绍与论文内容相关的国内外研究进展。最后一节应为“现有研究工作的不足”,以此表明论文工作的必要性、意义和价值。一般地,应控制在\textbf{2万字},约\textbf{15-20页},平均每页评述\textbf{15篇文献}左右。

本章写作要全面占有资料,实事求是地进行科学概括和分类,客观公正评价他人成果,发表自己意见要有理有据、见解独到。

写作方式:
\begin{itemize}
    \item 综述:针对拟研究的问题已有哪些工作和方法?
    \item 分析:在综述基础上述评已有方法优缺点。存在哪些缺陷?
    \item 归纳:方法分类,问题归纳(昭示本文的工作)。
    \item 动因:阐述自己的思路,进行问题分解。
\end{itemize}

写作禁忌:
\begin{itemize}
    \item 忌大——罗列已有的工作,面面俱到,没有自己的学术观点;
    \item 忌旧——追溯历史太久远,不适于计算机这个快速发展的学科;
    \item 忌空——涵盖范围太宽泛,没有快速收敛到论文需要解决的问题;
    \item 忌抄——完全照抄他人文献,缺乏自己的解读与评价。
\end{itemize}

参考文献的引用主要在这一章。根据学科的不同,参考文献引用格式也有所不同。完整的引用分为两部分:一部分是在论文原文中标注引用文献的出处(In-Text Citation),这部分很关键,如果直接引用或转述文献中的内容不当,或者没有标注来源,可能会被判为抄袭!另一部分是在论文的最后一部分“参考文献(References)”中罗列出的文献,论文中引用的所有资料都需要列出来。参考文献原则上应逐篇应用,不要一次引用多篇。
